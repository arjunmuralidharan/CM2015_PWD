%!TEX options = "--shell-escape"

\input{../../../Templates/preamble} %Adjust this based on where your Summary is stored
\title{CM2015: Programming with Data \\ Summary}
\author{Arjun Muralidharan}
\begin{document}
\input{../../../Templates/cover}

\section{Setting up your programming environment}
\subsection{Jupyter Setup} % (fold)
\label{ssub:jupyter_setup}

% subsubsection jupyter_setup (end)

Install Jupyter Lab using the following command.

\begin{minted}{shell}
	pip3 install jupyterlab
\end{minted}

Run Jupyter Lab using the following command.

\begin{minted}{shell}
	jupyter lab
\end{minted}

The interface allows you to create individual \texttt{.py} files, or entire \emph{notebooks} that contain \emph{cells}. Each cell is a small set of commands to execute and view the output of inline. A cell can contain code or markdown.

\subsection{Python Basics} % (fold)
\label{ssub:python_basics}

 Python relies on whitespace indentation, and comments are given using the \texttt{\#} sign. Code blocks are initiated with a colon, after which the contents of the block are indented one level.

 \begin{minted}{python}
 	# This is a simple Python script
 	for x in array:
 		if x < pivot:
 			less.append(x)
		else:
			greater.append(x)
 \end{minted}

 Everything in Python is an \emph{object}, making methods available on e.g.\ variables, strings, data structured, modules or even functions. In a notebook, pressing ``Tab'' will expose the available methods.

 Python assigns variable by reference, i.e.\ it uses binding. This means that when a variable \( a \) is assigned to another variable \( b \), changes to \( a \) will also reflect in \( b \).

 Python is \emph{strongly typed} but will make implicit type conversions when obvious. To find the type of a variable, use the following function.

 \begin{minted}{python}
 	isinstance(a, int)
 	isinstance(a, (int, float))

 	# To check the type in general
 	type(a)
 \end{minted}
This returns \texttt{true} if the provided variable is of the given type. In the second example, a tuple is used, where the function checks if the variable is of any type provided in the tuple.

Check if an object is iterable:
 \begin{minted}{python}
 	# Returns true if the argument is iterable
 	isiterable(a)

 	# Example: Convert an iterable structure to a list
 	if not isinstance(x, list) and isiterable(x):
 		x = list(x)
 \end{minted}


Python can import other Python files as modules, and even use variables from those files with new names.
 \begin{minted}{python}
	import some_module as sm
	from sm import PI as pi, g as gf

	r1 = sm.f(pi)
	r2 = gf(6, pi)
 \end{minted}

Python supports all the standard arithemtic and logical operators. One operator worth highlighting is the \emph{is} operator.

 \begin{minted}{python}
 	# Returns true if both variables are referencing the same object
 	a is b
 	a is not b

 	# NOTE: These are not the same as
 	a == b
 	a != b
 \end{minted}

The standard scalar types are \texttt{int}, \texttt{float}, \texttt{str}, \texttt{bytes}, \texttt{bool}, and \texttt{None}. These can also be used as functions to cast a variable to the specific type, if implicit conversion is possible.

While strings can be written with single or double quotes, in Python we can use triple quotes for strings spanning multiple lines.

 \begin{minted}{python}
	c = '''
	This is a multi-line
	string that has a lot
	of text in it
	'''
 \end{minted}

 Python string literals are immutable, you cannot modify a string directly using assignment to an index of the string. They do however behave like lists and can be iterated on.

 Strings also support a Python feature called templating, where templates are inserted into a string and then filled using the \texttt{format} method.

  \begin{minted}{python}
  # .2f represents a 2-decimal float, s is a string and d is an integer
  template = "{0:.2f} {1:s} are worth US${2:d}"
  template.format(4.5560, 'Argentine Pesos', 1)
 \end{minted}

 More details on templating are available in the Python documentation.

 \subsection{Control Flow}

 Conditional statements in Python.

  \begin{minted}{python}
if x < 0:
	print("Smaller")
elif x > 0
	print("Larger")
else
	pass
 \end{minted}

The \texttt{pass} statement moves forward in the code without executing anything, similar to a \texttt{continue} statement in other languages.

For loops always use a range or an iterator. While loops take a conditional.

 \begin{minted}{python}
 	for x in some_list:
 		# do smart stuff

 	while x < 0:
 		# do other smart stuff
 	 \end{minted}

 The \texttt{range} function returns an evenly spaced iterator of integers.

 \begin{minted}{python}
	list(range(6))
	# Returns a list [0, 1, 2, 3, 4, 5]
 \end{minted}

Ternary operators allow for more concise conditional statements.
 \begin{minted}{python}
 value = true_expr if condition else false_expr
 \end{minted}



% subsubsection python_basics (end)

\section{Variables, control flowand functions}
\section{Data structures}
\section{Reading and writing data on the filesystem}
\section{Retrieving data from the web}
\section{Retrieving data from databases using query languages}
\section{Cleaning and restructuring data}
\section{Data plotting}
\section{Version control systems}

\end{document}

